\subsection{Programmatūras izstrādes valoda un esošu bibliotēku izmantošana}

Par programmatūra izstrādes valodu tika izvēlēta JAVA programmēšanas valoda. Galvenie kritēriji kādēļ tika izvēlēta šī valoda bija: programmatūras spēja darboties uz dažādām platformām, esošu bibliotēku izmantošanas ērtība, bezmaksas izstrādes rīku pieejamība, kā arī projektā iesaistīto cilvēku pieredze ar šo valodu.

Lai atvieglotu programmatūras izstrādi tika meklētas esošas bibliotēkas, kas spētu veikt daļu no nepieciešamajiem uzdevumiem. Kā prasība šīm bibliotēkām tika izvirzīta tas, ka šīm bibliotēkām jābūt tīrai JAVA implementācijai, lai nebūtu problēmu uzstādīšanai uz dažādām platformām.

Datu ieguvei parasti tiek izmantoti DICOM standarta rīki. Ir vairākas bibliotēkas, kas piedāvā DICOM standarta rīkus, kuru salīdzinājums atrodams šeit \cite{dicombibcomp}. Mums bija pieejami tomogrāfijas uzņēmumi no VESSEL12 sacensībām kuri bija pieejami MetaImage formātā, kas sastāv no diviem failiem: teksta faila, kas satur aprakstošo informāciju pat datortomogrāfijas uzņēmumu (paplaīsnājums .mhd) un bināru datu failu (paplašinājums .raw). Diemžēl nevienā no pieejamām bibliotēkām, kurai būtu pieejama laba dokumentācija, funkcionalitāte šo failu nolasīšanai netika atrasta, tādēļ šī daļa tika implementēta patstāvīgi.

Asinsvadu segmentēšanas daļai bija nepieciešama lineārās algebras bibliotēka. Galvenās prasības šai bibliotēkai bija ātrdarbība, matricu operāciju pieejamība, laba matricas datu struktūras implementācija ar funkcionalitāti ērtai un elastīgai matricu elementu datu uzstādīšanai un piekļūšanai, kā arī labas dokumentācijas esamība. Populārāko lineārās algebras bibliotēku ātrdarbības salīdzinājums ir salīdzināts šiet - \cite{linearlibcomp}. Vadoties pēc iepriekš minētajiem kritērijiem tika izvēlēta EJML\cite{ejml} bibliotēka.

Asinsvadu segmentēšanas algoritmam bija nepieciešamas arī vairākas datorredzes pielietojumiem specifiskas attēlu apstrādes operācijas: 2D konvolūcija, Gausa attēlu piramīdas aprēķināšana, attēlu izmēru maiņa ar bikubisko interpolāciju un integrālattēla iegūšana. Šīs operācijas parasti atrodamas standarta datorredzes nozares bibliotēkās. Kā populārāko var minēt OpenCV\cite{opencv}, kas ir atvērtā koda bibliotēka ar visai labu atbalstu, un pieejamu interfeisu vairākām programmēšanas valodām: C++, Java, Python, Matlab. Kaut arī bibliotēkai ir pieejams interfeiss Java programmēšanas valodai, bibliotēkas pirmkods ir izstrādāts C++ valodā un, lai lietotu šo bibliotēku jāveic platformas atkarīgs bibliotēkas būvēšanas process un tā kā sākotnēji izvirzījām uzstādījumu izmantot tīras Java bibliotēkas, tad izvēlējāmies šo bibliotēku nelietot un šīs operācijas tika implementētas patstāvīgi.

Tā kā implementētais asinsvadu segmentēšanas algoritms bija visai lēns (~8 sekundēm vienam slāņa apstrādei), tika apskatīta iespēja izmantot OpenCV bibliotēku ar CUDA atbalstu, kas nodrošina vairāku attēlu apstrādes operāciju izpildīšanu uz grafiskās kartes, kas ļauj iegūt ātrāku operāciju izpildi (beigu implementācijā apstrādes laiks samazinājās līdz ~2.3 sekundēm vienam slānim). Šim nolūkam asinsvadu segmentēšanas algoritms tika implementēts C++ valodā izmantojot OpenCV bibliotēkas funkcijas attēlu apstrādes operācijām, kuras tiek izpildītas uz grafiskās kartes. Lai tiktu izmantots iepriekš izstrādātās programmatūras daļas, tad izmantojot JNI (Java Native Interface), tika izveidot interfeiss, kas ļāva izsaukt no Java programmas izsaukt C++ valodā implementēto asinsvadu segmentēšanas algoritmu.

Asinsvadu tīklojuma 3D ģeometrijas konstruēšanai tika izmantot maršējošo kubu algoritms, kuram gatavas implementācijas Java valodā ar labu dokumentāciju atrasta netika, tādēļ šis algoritms tika pašu implementēts.

Grafiskās lietotāja saskarnes izveidei tika izmantota Java standarta bibliotēkas platforma JavaFX, kas nodrošina arī 3D vizualizāciju.