\subsection{Programmatūras veicamie uzdevumi}

Rentgenu datortomogrāfijas uzņēmumi ir vairāku pelēktoņu 2D attēlu kopums, kur katrs no attēliem reprezentē vienu izmeklējamā objekta slāni. Kaut arī tomogrāfijas uzņēmumi satur 3D informāciju, tie tiešā veidā nav piemēroti 3D attēlošanai (3D ekrāni, 3D objektu 2D projekciju attēlošana 2D ekrānos). Šī iemesla dēļ ir izstrādāta programmatūra, kuras galvenais uzdevums ir: izmantojot datortomogrāfijas uzņēmumos esošo 3D informāciju, rekonstruēt plaušu asinsvadu tīkla 3D ģeometriju un saglabāt to  standartizētā 3D ģeometrijas failu formātā. 
Šo uzdevumu var sadalīt sekojošos apakšuzdevumos:
\begin{itemize}
\item{} Datu ieguve no datortomogrāfijas uzņēmumu failiem (.mhd, .raw);
\item{} Plaušas asinsvadu segmentēšana datortomogrāfijas uzņēmuma slāņos; 
\item{} Plaušas asinsvadu 3D ģeometrijas rekonstruēšana;
\item{} Plaušas asinsvadu 3D ģeometrijas saglabāšana standart formāta failos (Wavefront .obj);
\item{} Datortomogrāfijas slāņu 2D vizualizācija;
\item{} Rekonstruēto asinsvadu 3D vizualizācija;
\item{} Grafiskās lietotāja saskarnes izveide.
\end{itemize}

